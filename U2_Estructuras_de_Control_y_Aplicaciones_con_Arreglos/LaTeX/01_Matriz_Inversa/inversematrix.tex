\documentclass{article}
\def\InvDA{
$A^{-1}=\frac{1}{\mbox{\rm det\ }\{A\}}\mbox{\rm adj}\{A\}$
}
\def\InvD#1{
#1^{-1}=\frac{1}{\mbox{\rm det\ }\{#1\}}\mbox{\rm adj}\{#1\}
}

\begin{document}
\section{Def\/inici\'on de Matriz Inversa}
Sea $A\in R^{n\times n}$. Si $n=1$ $\Rightarrow$ 
$A^{-1}=\frac{1}{a_{00}}$, $a_{00}\neq 0$. Si $n\in\{2,3,4,5,\ldots\}$
$\Rightarrow$ 
\[
A^{-1}=\frac{1}
{\mbox{\rm det\ }\{A\}}\mbox{\rm adj}\{A\}
\]
\[
\InvD{B}
\]
%\InvDA
\section{Submatrices de una matriz}
En la definici\'{o}n del determinante de una matriz se utilizar\'{a}n las submatrices 
obtenidas cuando se eliminan una fila y una columna. Por comodidad, en este documento 
se describir\'{a}n \'{u}nicamente submatrices obtenidas eliminando la fila 0 y las 
columnas de la 0 a la $n-1$. Para una matriz $A\in\,R^{n\times n}$, se denotar\'{a} con 
$A_{0j}$ a la submatriz obtenida eliminando la  fila 0 y la columna  $j$ (para 
$j\in\{0,1,\ldots,n-1\}$). Por ejemplo, dada la matriz
\[
A = \left[\begin{array}{ccc}
1.00&2.00&3.00\\
4.00&40.41&42.43\\
5.00&44.45&46.47
\end{array}\right]
\]
Las submatrices $A_{00}$, $A_{01}$, y $A_{02}$ son las siguientes:
\[
A_{00} =  
\left[\begin{array}{cc}
40.41&42.43\\
44.45&46.47
\end{array}\right]
\]
\[
A_{01} =  
\left[\begin{array}{cc}
4.00&42.43\\
5.00&46.47
\end{array}\right]
\]
\[
A_{02} =  
\left[\begin{array}{cc}
4.00&40.41\\
5.00&44.45
\end{array}\right]
\]
Para ver un programa de ejemplo, en el que se def\/ine una funci\'{o}n que permite 
obtener los  elementos de una submatriz eliminando la fila 0 y las columnas 
$0,1,\ldots,n-1$, v\'{e}ase la forma de utilizar la funci\'{o}n 
\begin{verbatim} 
float** sub_matrix_(struct matriz *M,int colAElim)
{
      int i,j;
      float** R=(float**)malloc((M->m-1)*sizeof(float*));
      for(i=0;i < M->m-1;++i)
        R[i]=(float*)malloc((M->m-1)*sizeof(float));
      for(i=1;i < M->m;++i)
        for(j=0;j < M->n;++j){
          if(j<colAElim){
            R[i-1][j]=M->A[i][j];
          }
          if(j>colAElim){
            R[i-1][j-1]=M->A[i][j];
        }
     }
     return R;
}
\end{verbatim}
en el archivo 
\begin{verbatim}
https://github.com/programacionestructurada/2021-2/202110/
ProgEst02/00_07_Submatrices/Submatrices.c
\end{verbatim}
\section{Determinante de una matriz}
Dada una matriz $A\in R^{n\times n}$ se define su determinante con la siguiente 
f\'{o}rmula recursiva:
\[
\mbox{det}(A) = \left\{
\begin{array}{l}    
a_{00}\mbox{\rm\ \ \ ,  si }n = 1\\
a_{00}(-1)^{0+0}\mbox{det}(A_{00})+a_{01}(-1)^{0+1}\mbox{det}(A_{01})+\cdots\\
+ a_{0,n-1}(-1)^{0+n-1}\mbox{det}(A_{0,n-1})\mbox{\rm\ \ \ ,  si }n > 1
\end{array}
\right.
\]
Equivalentemente
\[
\mbox{det}(A) = \left\{
\begin{array}{l}    
a_{00}\mbox{\rm\ \ \ ,  si }n = 1\\
\sum_{j=0}^{n-1}a_{0j}(-1)^{0+j}\mbox{det}(A_{0j})\mbox{\rm\ \ \ ,  si }n > 1
\end{array}
\right.
\]
\end{document}
